\documentclass{article}

% if you need to pass options to natbib, use, e.g.:
% \PassOptionsToPackage{numbers, compress}{natbib}
% before loading nips_2017
%
% to avoid loading the natbib package, add option nonatbib:
% \usepackage[nonatbib]{nips_2017}

\usepackage[final]{nips_2017}


\usepackage[utf8]{inputenc} % allow utf-8 input
\usepackage[T1]{fontenc}    % use 8-bit T1 fonts
\usepackage{hyperref}       % hyperlinks
\usepackage{url}            % simple URL typesetting
\usepackage{booktabs}       % professional-quality tables
\usepackage{amsfonts}       % blackboard math symbols
\usepackage{nicefrac}       % compact symbols for 1/2, etc.
\usepackage{microtype}      % microtypography

% Choose a title for your submission
\title{Your title here}


\author{Student 1 \qquad Student 2 \qquad Student 3}

\begin{document}
% \nipsfinalcopy is no longer used

\maketitle

% We do not requrire you to write an abstract. Still, if you feel like it, please do so.
%\begin{abstract}
%\end{abstract}

Feel free to add more sections but those listed here are strongly recommended.
\section{Introduction}
You can keep this short. Ideally you introduce the task already in a way that highlights the difficulties  your method will tackle.

Some problems
\begin{itemize}
    \item Training data has no wrong endings = need to create wrong endings or only use validation set (option used by other papers)
            BUT we decided to generate false endings because using validation dataset to train is prone to verfitting
    \item Need to take into account sentiment, structure, etc.
\end{itemize}

\section{Methodology}
Your idea. You can rename this section if you like. Early on in this section -- but not necessarily first -- make clear what category your method falls into: Is it generative? Discriminative? Is there a particular additional data source you want to use?

\begin{itemize}
    \item Preprocessing: Created pos tagging
    \item Training data: Create wrong story endings in training data by
                            1) Taking right ending and replacing all nouns, verbs, adjectives and adverbs by words of the same category from the context, with a certain probability depending on the pos tag\\
                            Repeated operation until at least one word was changed (make sure sentences were different)
                            2) Taking sentence from context and replacing some words with a certain probability depending on the pos tag, with words from context.\\
                            Again, Repeated operation until at least one word was changed
                            Why? Structure of wrong sentence is different from structure of right sentence
    \item Tried several models: CNN, LSTM, Siamese LSTM...


\end{itemize}

\section{Model}
The math/architecture of your model. This should formally describe your idea from above. If you really want to, you can merge the two sections.

\section{Training}
What is your objective? How do you optimize it?

\section{Experiments}
This {\bf must} at least include the accuracy of your method on the validation set.

\section{Conclusion}
You can keep this short, too.

%addded
\section{Bibliography}
Bird, Steven, Edward Loper and Ewan Klein (2009), Natural Language Processing with Python. O’Reilly Media Inc.
\end{document}
